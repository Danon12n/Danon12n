\chapter{Ответы на вопросы}


\hspace{0cm} \textbf{1. Какие способы тестирования программы, кроме указанного в п.2, можете предложить еще?}

Для тестирования данной программы можно сравнить результаты работы методов разной точности, например, усовершенствованный метод Эйлера, который имеет 2-й порядок точности и меньшую погрешность, чем метод Эйлера 1-го порядка.

Так же для решения поставленной задачи можно воспользоваться приближенным аналитическим методом, например, методом Пикара.
\bigskip\bigskip\bigskip

\hspace{0cm} \textbf{2. Получите систему разностных уравнений для решения сформулированной задачи неявным методом трапеции. Опишите алгоритм реализации полученных уравнений.}

Рассмотрим выражение $u'(x) = f(x)$

Проинтегрируем обе части от $x_n$ до $x_{n+1}$:

\begin{equation*}
    \int\limits_{x_n}^{x_{n+1}}\frac{du}{dx}dx = \int\limits_{x_n}^{x_{n+1}}f(x)dx
\end{equation*} \

Взяв интеграл, получим:

\begin{equation*}
    u_n+1 - u_n = \int\limits_{x_n}^{x_{n+1}}f(x)dx 
\end{equation*}

\begin{equation*}
    u_n+1 = u_n + \int\limits_{x_n}^{x_{n+1}}f(x)dx
\end{equation*}

Вычислим интеграл методом трапеций:

\begin{equation*}
    y_n+1 = y_n + h * (\frac{f(x_n) + f(x_{n+1})}{2})
\end{equation*}

Применим данный метод к нашей задаче.
По условию задана система уравнений:

\begin{equation*}
    \begin{cases}
        \frac{dI}{dT} = \frac{U - (R_k + R_p (I))I}{L_k}\\
        \frac{dU}{dt} = -\frac{I}{C_k}
    \end{cases}
\end{equation*} \\

Начальные условия: $t=0,I=I_0,U_c=U_0$.

Введём обозначения: $\frac{dI}{dT}\equiv f_1 (I,U);  \frac{dU}{dt}\equiv f_2 (I)$. 

Воспользуемся методом трапеций, получим:

\begin{equation*}
    \begin{cases}
        I_{n+1} = I_n + h(\frac{f_1(I_n,U_n) + f_1(I_{n+1}, U_{n+1})}{2})\\
        U_{n+1} = U_n + h(\frac{f_2(I_n) + f_2(I_{n+1})}{2})\\
        t = 0\\
        I = I_0\\
        U = U_0
    \end{cases}
\end{equation*}

Подставим выражения $f_1(I,U)$ и $f_2(I)$:

\begin{equation*}
    I_{n+1} = I_n + h(\frac{U_n - (R_k + R_p(I_n))I_n + U_{n+1} - (R_k + R_p(I_{n+1}))I_{n+1}}{2 L_k })
\end{equation*}

\begin{equation*}
    U_{n+1} = U_n - h(\frac{I_n + I_{n+1}}{2C_k})
\end{equation*}

Подставим выражение для $U_{n+1}$ из второго уравнения в первое и решим относительно $I_{n+1}$. Получится выражение, которое решается методом простой итерации. Зная начальные условия и подставив их, мы можем получить $I_1$, которое затем можно подставить в (2), чтобы получить $U_1$. Последующие приближения можно получить таким же образом. Этот процесс продолжаем, пока не будет достигнута необходимая точность, то есть пока $|I_{n+1}-I_n |>\epsilon$.

\newpage

\hspace{0cm} \textbf{3. Из каких соображений проводится выбор численного метода того или иного порядка точности, учитывая, что чем выше порядок точности метода, тем он более сложен и требует, как правило, больших ресурсов вычислительной системы?}

Для метод четвёртого порядка точности должны быть четвёртые производные ограниченные, то есть правая часть должна быть непрерывна и ограничена вместе со своими четвёртыми производными. Если это не так, то метод четвёртого порядка точности не обеспечивает этот порядок.
Для метода второго порядка правая часть должна быть непрерывна и ограничена вместе со своими производными до второго порядка. Если это не так, то метод второго порядка точности не обеспечивают этот порядок и следует использовать метод Эйлера.

\bigskip\bigskip\bigskip

\hspace{0cm} \textbf{4. Можно ли метод Рунге-Кутта применить для решения задачи, в которой часть условий задана на одной границе, а часть на другой? Например, напряжение по-прежнему задано при $t = 0$, т.е. $t = 0$, $U = U0$, а ток задан в другой момент времени, к примеру, в конце импульса, т.е. при $t = T $, $I = IT $. Какой можете предложить алгоритм вычислений?}

Можно, только сначала нужно методом стрельбы свести эту задачу к задаче Коши для заданной системы уравнений, затем применить метод Рунге-Кутта. В заданном примере нужно найти значение напряжения в момент времени $t = T$, такое чтобы выполнялось краевое условие для тока $I = IT$.

