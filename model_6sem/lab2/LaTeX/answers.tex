\chapter{Ответы на вопросы}


\hspace{0cm} \textbf{1. Какие способы тестирования программы, кроме указанного в п.2, можете предложить еще?}

Для тестирования данной программы можно сравнить результаты работы методов разной точности, например, усовершенствованный метод Эйлера, который имеет 2-й порядок точности и меньшую погрешность, чем метод Эйлера 1-го порядка.

Так же для решения поставленной задачи можно воспользоваться приближенным аналитическим методом, например, методом Пикара.
\bigskip\bigskip\bigskip

\hspace{0cm} \textbf{2. Получите систему разностных уравнений для решения сформулированной задачи неявным методом трапеции. Опишите алгоритм реализации полученных уравнений.}

dfgdfg
\bigskip\bigskip\bigskip

\hspace{0cm} \textbf{3. Из каких соображений проводится выбор численного метода того или иного порядка точности, учитывая, что чем выше порядок точности метода, тем он более сложен и требует, как правило, больших ресурсов вычислительной системы?}

dfgdfg
\bigskip\bigskip\bigskip

\hspace{0cm} \textbf{4. Можно ли метод Рунге-Кутта применить для решения задачи, в которой часть условий задана на одной границе, а часть на другой? Например, напряжение по-прежнему задано при $t = 0$, т.е. $t = 0$, $U = U0$, а ток задан в другой момент времени, к примеру, в конце импульса, т.е. при $t = T $, $I = IT $. Какой можете предложить алгоритм вычислений?}

