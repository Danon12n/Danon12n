\chapter{Условия и задачи лабораторной работы}
 	
 	\hspace{0cm} \textbf{Тема лабораторной работы работы:} 
 	Программно-алгоритмическая реализация метода Рунге-Кутта 4-го порядка точности при решении системы ОДУ в задаче Коши.
 	
 	\hspace{0cm} \textbf{Цель лабораторной работы:} Получение навыков разработки алгоритмов решения задачи Коши при реализации моделей, построенных на системе ОДУ, с использованием метода Рунге-Кутта 4-го порядка точности.
 	
 	\hspace{0cm} \textbf{Описание алгоритма:} Задана система электротехнических уравнений, описывающих разрядный контур, включающий постоянное активное сопротивление $R_k$, нелинейное сопротивление $R_p(I)$, зависящее от тока $I$, индуктивность $L_k$ и емкость $C_k$. \\
 	
 	 \begin{equation*}
        \begin{cases}
            \frac{dI}{dT} = \frac{U - (R_k + R_p (I))I}{L_k}\\
            \frac{dU}{dt} = -\frac{I}{C_k}
        \end{cases}
    \end{equation*} \\
    
    \hspace{0cm} \textbf{Начальные условия:}
    
    $t = 0, I = I_0, U = U_0.$
    Здесь I, U - ток и напряжение на конденсаторе.
    Сопротивление $R_p$ рассчитать по формуле: \\
    
    \begin{equation*}
         R_p = \frac{l_p}{2\pi R^2\int\limits_0^1\sigma(T(z))zdz}
    \end{equation*} \\
    
    Для функции $T(z)$ применить выражения $T(z) = T_0 + (T_w - T_0) z^m.$ Параметры $T_0, m$ находятся интерполяцией из таблицы 1 при известном токе I. Коэффициент электропроводности $\sigma(T)$ зависит от $T$ и рассчитывается интерполяцией из таблицы 2. \\
    
    \hspace{0cm} \textbf{Таблица 1} \\
    
    \begin{tabular}{ | c | c | c | }
    \hline
    I, A & $T_0, K$ & m  \\ \hline
    0.5 & 6730 & 0.50 \\
    1 & 6790 & 0.55 \\
    5 & 7150 & 1.7 \\
    10 & 7270 & 3 \\
    50 & 8010 & 11 \\
    200 & 9185 & 32 \\
    400 & 10010 & 40 \\
    800 & 11140 & 41 \\
    1200 & 12010 & 39 \\
    \hline
    \end{tabular}  
    
    \newpage
    
    \hspace{0cm} \textbf{Таблица 2} \\
    
    \begin{tabular}{ | c | c | }
    \hline
    T, K & $\sigma$ 1/Ом см  \\ \hline
    4000 & 0.031 \\
    5000 & 0.27 \\
    6000 & 2.05 \\
    7000 & 6.06 \\
    8000 & 12.0 \\
    9000 & 19.9 \\
    10000 & 29.6 \\
    11000 & 41.1 \\
    12000 & 54.1 \\
    13000 & 67.7 \\
    14000 & 81.5 \\
    \hline
    \end{tabular} \\
    
    \hspace{0cm} \textbf{Параметры разрядного контура:} \\
    
    $R = 0.35 $ см \\
    $l_e = 12 $ см \\
    $L_k = 187*10^-6 $ Гн \\ 
    $C_k = 268*10^-6 $ Ф \\
    $R_k = 0.25 $ Ом \\
    $U_co = 1400 $ В \\
    $I_0 = 0..3 $ А \\
    $T_w = 2000 $ К \\
    
    \hspace{0cm} \textbf{Метод Рунге-Кутта 4-го порядка точности:} \\
    
    $y_n+1 = y_n + \frac{k1 + 2 * k_2 + 2 * k_3 + k_4}{6}$ \\
    
    $z_n+1 = z_n + \frac{p1 + 2 * p_2 + 2 * p_3 + p_4}{6}$ \\
    
    $k_1 = h_nf(x_n, y_n, z_n)$ \
    
    $k_2 = h_nf(x_n + \frac{h_n}{2}, y_n + \frac{k_1}{2}, z_n + \frac{p_1}{2})$ \
    
    $k_3 = h_nf(x_n + \frac{h_n}{2}, y_n + \frac{k_2}{2}, z_n + \frac{p_2}{2})$ \
    
    $k_4 = h_nf(x_n + h_n, y_n + k_3, z_n + p_3)$ \\
    
    $p_1 = h_n\varphi(x_n, y_n, z_n)$ \
    
    $p_2 = h_n\varphi(x_n + \frac{h_n}{2}, y_n + \frac{k_1}{2}, z_n + \frac{p_1}{2})$ \
    
    $p_3 = h_n\varphi(x_n + \frac{h_n}{2}, y_n + \frac{k_2}{2}, z_n + \frac{p_2}{2})$ \
    
    $p_4 = h_n\varphi(x_n + h_n, y_n + k_3, z_n + p_3)$ \\
    
    
    