\chapter{Численные и явные методы}
    
    \section{Метод Пикара}
    Метод Пикара является представителем приближенных методов решения рассматриваемого класса задач. Идея метода сводится к процедуре последовательных приближений для решения интегрального уравнения, к которому приводится исходное дифференциальное уравнение.

    Поставлена задача Коши:
    \eq{
        \begin{cases}
            u'(x) = f(x, u(x)), \\
            u(x_0) = u_0
        \end{cases}
    }
    
    Проинтегрируем выписанное уравнение
    \eq{u(x) = u_0 + \int_{x_0}^{x} f(t,u(t)) \,dt \ }

    Процедура последовательных приближений метода Пикара реализуется согласно следующей схеме
    \eq{ y_s(x) = u_0 + \int_{x_0}^{x} f(t,y_{s-1}(t)) \,dt \ , }
    причем $y_0(t) = v_0$, (i – номер итерации).
    
    Заданная в лабораторной работе ОДУ, не имеющее аналитического решения
    \eq{
        \begin{cases}
            u'(x) = x^2 + u^2, \\
            u(0) = 0,
        \end{cases}
    }
    
    Правая часть непрерывна и удовлетворяет условию Липшица. Значит, решение существует, а метод Пикара сойдется. По схеме Пикара рассчитаем первые четыре приближения для заданного ОДУ.
    \eq{y_1(x) = 0 + \int_{0}^{x} t^2 \,dt \  = \frac{x^3}{3}}
    \eq{y_2(x) = 0 + \int_{0}^{x} (t^2  + (\frac{t^3}{3})^2) \,dt \  = \frac{x^3}{3} + \frac{x^7}{63}}
    \eq{y_3(x) = 0 + \int_{0}^{x} (t^2  + (\frac{t^3}{3} + \frac{t^7}{63})^2) \,dt \  = \frac{x^3}{3} + \frac{x^7}{63} + \frac{2x^{11}}{2079} + \frac{x^{15}}{59535}}
    \eq{\begin{gathered}
        	y_4(x) = 0 + \int_{0}^{x} (t^2  + (\frac{t^3}{3} + \frac{t^7}{63} + \frac{2t^{11}}{2079} + \frac{t^{15}}{59535})^2) \,dt \ = \\ = \frac{x^3}{3} + \frac{x^7}{63} + \frac{2x^{11}}{2079} + \frac{13x^{15}}{218295} + \frac{82x^{19}}{37328445} + \frac{662x^{23}}{10438212015} + \frac{4x^{27}}{3341878155} + \frac{x^{31}}{109876902975}
        \end{gathered}}
    
    \section{Метод Эйлера}
    Также задача может быть решена с помощью численных методов. 
    \eq{y_{n + 1} = y_n + hf(x_n, y_n)}
    \eq{f(x_n, y_n) = y_n^2 + x_n^2}
    
    
    \section{Метод Рунге-Кутта 2-ого порядка точности}
    \begin{equation}
    	y_{n+1} = y_n + h[(1 - \alpha) k_1 + \alpha k_2],
    	\label{eq_rungre:ref}
    \end{equation}
    где  
    \eq{k_1 = f(x_n, y_n),}
    \eq{k_2 = f(x_n + \frac{h}{2\alpha}, y_n + \frac{h}{2\alpha}k_1),}
    В практике расчетов используют формулу \eqref{eq_rungre:ref} при значениях $\alpha = 1$, $\alpha = \frac{1}{2}$. В лабораторной работе приняли $\alpha = \frac{1}{2}$

   \newpage