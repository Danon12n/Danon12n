\chapter{Условия и задачи лабораторной работы}
 	
 	\hspace{0cm} \textbf{Тема лабораторной работы работы:} Программная реализация приближенного аналитического метода и численных алгоритмов первого и второго порядков точности при решении задачи Коши для ОДУ.
 	
 	\hspace{0cm} \textbf{Цель лабораторной работы:} Получение навыков решения задачи Коши для ОДУ методами Пикара (первого, второго, третьего или четвертого порядка точности) и явными методами первого порядка точности (Эйлера) и второго порядка точности (Рунге-Кутта).

    \hspace{0cm} \textbf{Задача Коши:}
    
    \begin{equation*}
        \begin{cases}
            u'(x) = x^2 + u^2, \\
            u(0) = 0,
        \end{cases}
    \end{equation*}
    
    \hspace{0cm} Результатом работы программы должна быть таблица, содержащая значение агрумента и значения полученные методом Пикара, явным методом Эйлера и методом Рунге-Кутта.
    
    \hspace{0cm} \textbf{Вопросы по лабораторной работе} 
    
    \begin{enumerate}
    \item Укажите интервалы значений аргумента, в которых можно считать решением заданного уравнения каждое из первых 4-х приближений Пикара. Точность результата оценивать до второй цифры после запятой. Объяснить свой ответ.
    \item Пояснить, каким образом можно доказать правильность полученного результата при фиксированном значении аргумента в численных методах.
    \item Каково значение функции при $x=2$, т.е. привести значение $u(2)$.
\end{enumerate}

    \newpage