\usepackage[T2A]{fontenc}
\usepackage[utf8]{inputenc}
\usepackage[english,russian]{babel}

\usepackage[left=2.5cm, right=1.5cm, top=2.5cm, bottom=2.5cm]{geometry}

\usepackage[tableposition=top,singlelinecheck=false]{caption}
\usepackage{subcaption}

\DeclareCaptionLabelFormat{gostfigure}{Рисунок #2}
\DeclareCaptionLabelFormat{gosttable}{Таблица #2}
\DeclareCaptionLabelSeparator{gost}{~---~}
\captionsetup{labelsep=gost}
\captionsetup*[figure]{labelformat=gostfigure}
\captionsetup*[table]{labelformat=gosttable}
\renewcommand{\thesubfigure}{\asbuk{subfigure}}

\usepackage{microtype}
\sloppy

\usepackage{setspace}
\onehalfspacing

\usepackage{indentfirst}
\setlength\parindent{5ex}

\usepackage{titlesec}
\titleformat{\chapter}{\LARGE\bfseries}{\thechapter}{20pt}{\Large}
\titleformat{\section}{\Large\bfseries}{\thesection}{20pt}{\large}

\addto{\captionsrussian}{\renewcommand*{\contentsname}{Содержание}}
\usepackage[square]{natbib}
\renewcommand{\bibsection}{\chapter*{Список литературы}}

\makeatletter
\def\@biblabel#1{#1. }
\makeatother

\usepackage{caption}

\usepackage{wrapfig}
\usepackage{float}
\usepackage{multirow}

\usepackage{graphicx}
\newcommand{\imgwc}[4]
{
  \begin{figure}[#1]
    \center{\includegraphics[width=#2]{inc/img/#3}}
    \caption{#4}
    \label{img:#3}
  \end{figure}
}
\newcommand{\imghc}[4]
{
  \begin{figure}[#1]
    \center{\includegraphics[height=#2]{inc/img/#3}}
    \caption{#4}
    \label{img:#3}
  \end{figure}
}
\newcommand{\imgsc}[4]
{
  \begin{figure}[#1]
    \center{\includegraphics[scale=#2]{inc/img/#3}}
    \caption{#4}
    \label{img:#3}
  \end{figure}
}

\usepackage{pgfplots}
\pgfplotsset{compat=newest}

\usepackage{minted}
\newminted{csh}{fontsize=\small, fontfamily=rm}

\usepackage{listings}
\usepackage{listingsutf8}
\lstset{
  language=csh,
  basicstyle=\footnotesize\ttfamily,
  keywordstyle=\color{blue},
  stringstyle=\color{red},
  commentstyle=\color{gray},
  numbers=left,
  numberstyle=\tiny,
  numbersep=5pt,
  frame=false,
  breaklines=true,
  breakatwhitespace=true,
  inputencoding=utf8
}

\newcommand{\code}[1]{\texttt{#1}}

\usepackage{amsmath}
\usepackage{amssymb}

\usepackage[unicode]{hyperref}
\hypersetup{hidelinks}

\makeatletter
\newcommand{\vhrulefill}[1]
{
  \leavevmode\leaders\hrule\@height#1\hfill \kern\z@
}
\makeatother

